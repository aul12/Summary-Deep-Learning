\section{Convolutional Neural Networks}

\q{What is the difference between convolution and auto-correlation?}

\q{Which signal out of f 1,2 and g 1,2 is the kernel in the above figures? (Slide 8)}

\q{There will be some empty cells in the resulting signal. Is this a
problem?}

\q{What is the limitation of high dimensional convolutions?}

\q{What is the requirement of the data structure in order to use
convolutional layers?}

\q{Since we did not use padding, what happened with the resulting
feature map?}

\q{Why is it important to use multiple filters / kernels?}

\q{What happens in the forward and backward pass of a convolution
layer?}

\q{How do we learn the filter values?}

\q{Can we use the optimization techniques from the previous lectures?}

\q{$3 \times 3$ input, $2 \times 2$ filter, no padding: What will be the dimensions of
the resulting matrix?}

\q{Consider a $7 \times 7$ image, $3 \times 3$ filter, stride 3 and no padding strategy: What is the output size?}

\q{What do we have to change to avoid reducing the spatial
dimensions?}

\q{How do we need to pad in order to retain the data dimensions after a
$3\times 3$ convolution?}

\q{Where do we use depth-wise convolutions?}

\q{What is the difference of a $3 \times 3$ from a $5 \times 5$ kernel?}

\q{When do we use a $5 \times 5$ or $7 \times 7$ kernel?}

\q{How can we reduce the size of the input tensor using convolutional
kernel(s)?}

\q{Which of the two pooling operations mimics the human brain?}

\q{What are low- and mid-level features in image data?}




